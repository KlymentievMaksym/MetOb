% Звіт
% 1. Вихідні дані.

% 2. Письмове виконання допрограмового етапу, результатом якого повинні бути проміжки, щодо яких проводиться уточнення.

% 3. Лістинг програми уточнення коренів за методами бісекції, хорд, дотичних
% (вхідними даними для цієї програми є координати проміжків [ai, bi] та коефіцієнти поліному)
% та результати дії програми. На кожній ітерації методу слід виводити такі дані: номер ітерації, наближене значення кореня, критерій завершення ітерацій.
\documentclass{article}
\usepackage[utf8]{inputenc}
\usepackage[english, ukrainian]{babel}
\usepackage{fontsize}
\usepackage{geometry}
\usepackage{amsthm}
\usepackage{amsfonts}
\usepackage{graphicx}
\usepackage[ruled]{algorithm2e}
\usepackage{hyperref}
\usepackage{biblatex}
\usepackage{csquotes}
\usepackage{mathtools}
\usepackage{amsmath}
\usepackage{amssymb}
\usepackage{bbm}
\usepackage{tabularx}
\hypersetup{colorlinks=true, linkcolor=[RGB]{255, 3, 209}, citecolor={black}}

\graphicspath{ {../Images/} }

\begin{document}
    \begin{titlepage}
        \begin{center}
            \begin{center}
                НАЦІОНАЛЬНИЙ ТЕХНІЧНИЙ УНІВЕРСИТЕТ УКРАЇНИ
                «КИЇВСЬКИЙ ПОЛІТЕХНІЧНИЙ ІНСТИТУТ імені Ігоря СІКОРСЬКОГО»

                Фізико-технічний інститут
            \end{center}
        $\newline$
        \vspace{3.3cm}
        
        {КОМП’ЮТЕРНИЙ ПРАКТИКУМ № 1.\\РОЗВ’ЯЗАННЯ НЕЛIНIЙНИХ РIВНЯНЬ}
        \vspace{5cm}
        \begin{flushright}
            Виконав\\студент 3 курсу ФТІ\\групи ФІ-21\\Климентьєв Максим Андрійович
            
            \vspace{1cm}

            Перевірив:\\\underline{\hspace{5cm}}\\Оцінка:\\\underline{\hspace{5cm}}
        \end{flushright}
        \vspace{3cm}
        Київ --- 2025
        \end{center}
    \end{titlepage}
    \newpage

    \pagenumbering{gobble}
    \tableofcontents
    \cleardoublepage
    \pagenumbering{arabic}
    \setcounter{page}{3}

    \newpage
    \section{Вихідні дані.}
    \begin{tabular}{ |c|c| }
        \hline
        Варіант & Вигляд рівняння \\ 
        \hline
        10 & $ -2 \cdot x^4 + x^3 + 5 \cdot x^2 - 2 \cdot x + 7 = 0 $ \\ 
        \hline
    \end{tabular}

    \section{Письмове виконання допрограмового етапу, результатом якого повинні бути проміжки, щодо яких проводиться уточнення.}
    % \includegraphics{}
    % \includegraphics{}
    % \includegraphics{}
    \newpage
    \section{Лістинг програми уточнення коренів за методами бісекції, хорд, дотичних (вхідними даними для цієї програми є координати проміжків [ai, bi] та коефіцієнти поліному) та результати дії програми. На кожній ітерації методу слід виводити такі дані: номер ітерації, наближене значення кореня, критерій завершення ітерацій.}
        \textbf{} --- 
        
\end{document}