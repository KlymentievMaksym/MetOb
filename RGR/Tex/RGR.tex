% Звіт
% 1. Вихідні дані.

% 2. Письмове виконання допрограмового етапу, результатом якого повинні бути проміжки, щодо яких проводиться уточнення.

% 3. Лістинг програми уточнення коренів за методами бісекції, хорд, дотичних
% (вхідними даними для цієї програми є координати проміжків [ai, bi] та коефіцієнти поліному)
% та результати дії програми. На кожній ітерації методу слід виводити такі дані: номер ітерації, наближене значення кореня, критерій завершення ітерацій.

\documentclass{article}
\usepackage[utf8]{inputenc}
\usepackage[english, ukrainian]{babel}
\usepackage{fontsize}
\usepackage{geometry}
\usepackage{amsthm}
\usepackage{amsfonts}
\usepackage{graphicx}
\usepackage[ruled]{algorithm2e}
\usepackage{hyperref}
\usepackage{biblatex}
\usepackage{csquotes}
\usepackage{mathtools}
\usepackage{amsmath}
\usepackage{amssymb}
\usepackage{bbm}
\usepackage{tabularx}
\hypersetup{colorlinks=true, linkcolor=[RGB]{255, 3, 209}, citecolor={black}}

\graphicspath{ {../Images/} }

\begin{document}
    \begin{titlepage}
        \begin{center}
            \begin{center}
                НАЦІОНАЛЬНИЙ ТЕХНІЧНИЙ УНІВЕРСИТЕТ УКРАЇНИ
                «КИЇВСЬКИЙ ПОЛІТЕХНІЧНИЙ ІНСТИТУТ імені Ігоря СІКОРСЬКОГО»

                Фізико-технічний інститут
            \end{center}
        $\newline$
        \vspace{3.3cm}
        
        {
        РОЗРАХУНКОВО-ГРАФІЧНА РОБОТА
        
        з кредитного модуля «Методи обчислень»
        
        на тему:
        
        «ОБЧИСЛЮВАЛЬНЕ РОЗВ’ЯЗАННЯ ДИФЕРЕНЦІАЛЬНИХ
        
        РІВНЯНЬ У ЧАСТИННИХ ПОХІДНИХ»
        
        Варіант №10
        }
        \vspace{3cm}
        \begin{flushright}
            Виконав\\студент 3 курсу ФТІ\\групи ФІ-21\\Климентьєв Максим Андрійович
            
            \vspace{1cm}

            Перевірив:\\...\\Оцінка:\\...
        \end{flushright}
        \vspace{3.5cm}
        Київ --- 2025
        \end{center}
    \end{titlepage}
    \newpage

    \pagenumbering{gobble}
    \tableofcontents
    \cleardoublepage
    \pagenumbering{arabic}
    \setcounter{page}{3}

    \newpage
    \section{ПОСТАНОВКА ЗАДАЧІ}
    \textbf{Варіант 10}

    Знайти чисельний розв’язок рівняння коливань струни:

    $$ \frac{\partial^2{u}}{\partial{t^2}} = \frac{\partial^2{u}}{\partial{x^2}} + F(t, x) $$
    % $$ u_{tt} = u_{xx} + F(t, x) $$
    $$ 0 < x < L = 1 $$
    $$ u \vert_{(t = 0)} = u_0 = x \cdot (x+1) $$
    % $$ u(t = 0) = u_0 = x \cdot (x+1) $$
    $$ \frac{\partial{u}}{\partial{t}} \vert_{(t = 0)} = 0 $$
    % $$ \frac{\partial{u}}{\partial{t}}(t = 0) = 0 $$
    % $$ u\vert_{t = 0} = u_1(t) $$
    $$ u(t, 0) = u_1(t) $$
    % $$ u\vert_{t = L} = u_2(t) $$
    $$ u(t, L) = u_2(t) $$
    \hrule
    $$ u(t,x) = u_0(x) \cdot \cos(\pi \cdot t) $$
    $$ u_0(x) = u_0 = x \cdot (x+1) $$
    $$ u(t,x) = x \cdot (x+1) \cdot \cos(\pi \cdot t) $$
    $$ u(t,0) = 0 \cdot 1 \cdot \cos(\pi \cdot t) = 0 $$
    $$ u(t, L) = L \cdot (L+1) \cdot \cos(\pi \cdot t) = 1 \cdot 2 \cdot \cos(\pi \cdot t) = 2 \cdot \cos(\pi \cdot t) $$
    Навести приклади процесів, які моделюються за допомогою диференціальних рівнянь у частинних похідних гіперболічного типу
    
    \newpage
    \section{ОГЛЯД ТА АНАЛІЗ ІСНУЮЧИХ МЕТОДІВ ЧИСЕЛЬНОГО РОЗВ’ЯЗАННЯ ДРЧП}
    \newpage
    \section{ДОСЛІДЖЕННЯ УМОВ ЗАСТОСУВАННЯ ОБРАНОГО МЕТОДУ}

        Явна схема

        $$ \frac{u_i^{k+1} - 2 \cdot u_i^{k} + u_i^{k-1}}{\Delta{t}^2} = \frac{u_{i+1}^{k} - 2 \cdot u_i^{k} + u_{i-1}^{k}}{\Delta{x}^2} + F_i^{k} $$
        $$ u_i^{k+1} - 2 \cdot u_i^{k} + u_i^{k-1} = \Delta{t}^2 \cdot \left( \frac{u_{i+1}^{k} - 2 \cdot u_i^{k} + u_{i-1}^{k}}{\Delta{x}^2} + F_i^{k} \right) $$
        $$ u_i^{k+1} = 2 \cdot u_i^{k} - u_i^{k-1} + \Delta{t}^2 \cdot \left( \frac{u_{i+1}^{k} - 2 \cdot u_i^{k} + u_{i-1}^{k}}{\Delta{x}^2} + F_i^{k} \right) $$
        $$ u(i, k+1) = 2 \cdot u(i, k) - u(i, k-1) + \Delta{t}^2 \cdot \left( \frac{u(i+1, k) - 2 \cdot u(i, k) + u(i-1, k)}{\Delta{x}^2} + F(i, k) \right) $$

        Явно-неявна схема (тришарова схема з вагами)

        \begin{equation*}
            \begin{split}
                \frac{u_i^{k+1} - 2 \cdot u_i^{k} + u_i^{k-1}}{\Delta{t}^2} & = \sigma_1 \left( \frac{u_{i+1}^{k+1} - 2 \cdot u_i^{k+1} + u_{i-1}^{k+1}}{\Delta{x}^2} + F_i^{k+1} \right) +\\
                & + (1 - \sigma_1 - \sigma_2) \left( \frac{u_{i+1}^{k} - 2 \cdot u_i^{k} + u_{i-1}^{k}}{\Delta{x}^2} + F_i^{k} \right) +\\
                & + \sigma_2 \left( \frac{u_{i+1}^{k-1} - 2 \cdot u_i^{k-1} + u_{i-1}^{k-1}}{\Delta{x}^2} + F_i^{k-1} \right)\\
            \end{split}
        \end{equation*}

        $$ u \vert_{(t = 0)} = u_0 = x \cdot (x+1) $$
        $$ \frac{\partial{u}}{\partial{t}} \vert_{(t = 0)} = 0 $$
        % $$ \frac{u^1 - u^0}{\Delta{t}} = u_{10} \rightarrow u^1 $$
        $$ A\left(u^{k+1}\right) = B\left(u^k, u^{k-1}\right) \rightarrow \text{СЛАР або СНАР} $$

    \newpage
    \section{ОПИС ПРОГРАМНОЇ РЕАЛИЗАЦІЇ}
    \newpage
    \section{ОГЛЯД МЕТОДІВ ПІДВИЩЕННЯ ТОЧНОСТІ}
    \newpage
    \section{ЗАСТОСУВАННЯ МЕТОДУ ПІДВИЩЕННЯ ТОЧНОСТІ ТА ЕФЕКТИВНОСТІ РОЗВ’ЯЗКУ ДО ПРИКЛАДУ РОБОТИ}
    \newpage
    \section{ВИСНОВКИ}
    \newpage
    \section{СПИСОК ВИКОРИСТАНИХ ДЖЕРЕЛ}
    \begin{enumerate}
        \item 
    \end{enumerate}
    \newpage
    \section{ДОДАТКИ}
        % \includegraphics{}
        % \includegraphics{}
        % \includegraphics{}
        % \includegraphics{}
        % \includegraphics{}

\end{document}