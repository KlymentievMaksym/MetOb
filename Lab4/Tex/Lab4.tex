\documentclass{article}
\usepackage[utf8]{inputenc}
\usepackage[english, ukrainian]{babel}
\usepackage{fontsize}
\usepackage{geometry}
\usepackage{amsthm}
\usepackage{amsfonts}
\usepackage{graphicx}
\usepackage[ruled]{algorithm2e}
\usepackage{hyperref}
\usepackage{biblatex}
\usepackage{csquotes}
\usepackage{mathtools}
\usepackage{amsmath}
\usepackage{amssymb}
\usepackage{bbm}
\usepackage{tabularx}
\hypersetup{colorlinks=true, linkcolor=[RGB]{255, 3, 209}, citecolor={black}}

\graphicspath{ {../Images/} }

\begin{document}
    \begin{titlepage}
        \begin{center}
            \begin{center}
                НАЦІОНАЛЬНИЙ ТЕХНІЧНИЙ УНІВЕРСИТЕТ УКРАЇНИ
                «КИЇВСЬКИЙ ПОЛІТЕХНІЧНИЙ ІНСТИТУТ імені Ігоря СІКОРСЬКОГО»

                Фізико-технічний інститут
            \end{center}
        $\newline$
        \vspace{3.3cm}
        
        {КОМП’ЮТЕРНИЙ ПРАКТИКУМ № 4.\\ОБЧИСЛЕННЯ ВЛАСНИХ ЗНАЧЕНЬ}
        \vspace{5cm}
        \begin{flushright}
            Виконав\\студент 3 курсу ФТІ\\групи ФІ-21\\Климентьєв Максим Андрійович
            
            \vspace{1cm}

            Перевірив:\\\underline{\hspace{5cm}}\\Оцінка:\\\underline{\hspace{5cm}}
        \end{flushright}
        \vspace{3cm}
        Київ --- 2025
        \end{center}
    \end{titlepage}
    \newpage

    \pagenumbering{gobble}
    \tableofcontents
    \cleardoublepage
    \pagenumbering{arabic}
    \setcounter{page}{3}

    \newpage
    \section{ПОСТАНОВКА ЗАДАЧІ}
    Для методу Данилевського: привести матрицю до вигляду Фробеніуса, розв’язати отриману систему за допомогою методу із практикуму 2 або 3, отримати коефіцієнти характеристичного рівняння. Розв’язати характеристичне рівняння за допомогою одного з методів із практикуму 1 і отримати власні числа.
    \section{Вихідна система}
    \begin{tabular}{ |c|c|c| }
        \hline
        Варіант & Матриця & Метод \\ 
        \hline
        10
        &
        $\begin{matrix}
            6.20 & 1.10 & 0.94 & 1.21\\
            1.10 & 4.10 & 1.30 & 0.16\\
            0.94 & 1.30 & 7.40 & 1.10\\
            1.21 & 0.16 & 1.10 & 9.10
        \end{matrix}$
        &
        Данілевського
        \\ 
        \hline
    \end{tabular}

% Звіт має містити:
% - для методу Данилевського: матриці iM та 1−
% iM ; результуючу матрицю у формі
% Фробеніуса, отримане характеристичне рівняння, власні числа.}
% Для всіх варіантів у звіті потрібно навести перевірку у Маthcad чи Matlab
\end{document}