\documentclass{article}
\usepackage[utf8]{inputenc}
\usepackage[english, ukrainian]{babel}
\usepackage{fontsize}
\usepackage{geometry}
\usepackage{amsthm}
\usepackage{amsfonts}
\usepackage{graphicx}
\usepackage[ruled]{algorithm2e}
\usepackage{hyperref}
\usepackage{biblatex}
\usepackage{csquotes}
\usepackage{mathtools}
\usepackage{amsmath}
\usepackage{amssymb}
\usepackage{bbm}
\usepackage{tabularx}
\hypersetup{colorlinks=true, linkcolor=[RGB]{255, 3, 209}, citecolor={black}}

\graphicspath{ {../Images/} }

\begin{document}
    \begin{titlepage}
        \begin{center}
            \begin{center}
                НАЦІОНАЛЬНИЙ ТЕХНІЧНИЙ УНІВЕРСИТЕТ УКРАЇНИ
                «КИЇВСЬКИЙ ПОЛІТЕХНІЧНИЙ ІНСТИТУТ імені Ігоря СІКОРСЬКОГО»

                Фізико-технічний інститут
            \end{center}
        $\newline$
        \vspace{3.3cm}
        
        {
        РОЗРАХУНКОВО-ГРАФІЧНА РОБОТА
        
        з кредитного модуля «Методи обчислень»
        
        на тему:
        
        «ОБЧИСЛЮВАЛЬНЕ РОЗВ’ЯЗАННЯ ДИФЕРЕНЦІАЛЬНИХ
        
        РІВНЯНЬ У ЧАСТИННИХ ПОХІДНИХ»
        
        Варіант №10
        }
        \vspace{3cm}
        \begin{flushright}
            Виконав\\студент 3 курсу ФТІ\\групи ФІ-21\\Климентьєв Максим Андрійович
            
            \vspace{1cm}

            Перевірив:\\\underline{\hspace{5cm}}\\Оцінка:\\\underline{\hspace{5cm}}
        \end{flushright}
        \vspace{3.5cm}
        Київ --- 2025
        \end{center}
    \end{titlepage}
    \newpage

    \pagenumbering{gobble}
    \tableofcontents
    \cleardoublepage
    \pagenumbering{arabic}
    \setcounter{page}{3}

    \newpage
    \section{ПОСТАНОВКА ЗАДАЧІ}
    \textbf{Варіант 10}

    Знайти чисельний розв’язок рівняння коливань струни:

    $$ \frac{\partial^2{u}}{\partial{t^2}} = \frac{\partial^2{u}}{\partial{x^2}} + F(t, x) $$
    $$ 0 < x < L = 1 $$
    $$ u(t = 0) = u_0 = x \cdot (x+1) $$
    $$ \frac{\partial{u}}{\partial{t}}(t = 0) = 0 $$
    $$ u(t, 0) = u_1(t) $$
    $$ u(t, L) = u_2(t) $$

    \hrule

    $$ u(t,x) = u_0(x) \cdot \cos(\pi \cdot t) $$
    $$ u_0(x) = u_0 = x \cdot (x+1) $$
    $$ u(t,x) = x \cdot (x+1) \cdot \cos(\pi \cdot t) $$

    \hrule

    $$ u(t,0) = 0 \cdot 1 \cdot \cos(\pi \cdot t) = 0 $$
    $$ u(t, L) = L \cdot (L+1) \cdot \cos(\pi \cdot t) = 1 \cdot 2 \cdot \cos(\pi \cdot t) = 2 \cdot \cos(\pi \cdot t) $$

    \hrule

    $$ \frac{\partial{u}}{\partial{x}} = 2 \cdot x \cdot \cos(\pi \cdot t) + \cos(\pi \cdot t) $$
    $$ \frac{\partial^2{u}}{\partial{x^2}} = 2 \cdot \cos(\pi \cdot t) $$

    $$ \frac{\partial{u}}{\partial{t}} = -\pi \cdot x \cdot (x+1) \cdot \sin(\pi \cdot t) $$
    % $$ \frac{\partial{u}}{\partial{t}} = -\pi \cdot x \cdot (x+1) \cdot \sin(\pi \cdot t) $$
    $$ \frac{\partial^2{u}}{\partial{t^2}} = -\pi^2 \cdot x \cdot (x+1) \cdot \cos(\pi \cdot t) $$

    \hrule

    $$ \frac{\partial^2{u}}{\partial{t^2}} = \frac{\partial^2{u}}{\partial{x^2}} + F(t, x) $$
    $$ \frac{\partial^2{u}}{\partial{t^2}} - \frac{\partial^2{u}}{\partial{x^2}} = F(t, x) $$
    $$ F(t, x) = -\pi^2 \cdot x \cdot (x+1) \cdot \cos(\pi \cdot t) - 2 \cdot \cos(\pi \cdot t) $$
    $$ F(t, x) = -\cos(\pi \cdot t) \cdot \left(\pi^2 \cdot x \cdot (x+1) + 2\right) $$

    \hrule

    Рівняння --- гіперболічного типу.
    
    Є нелінійність - будемо робити через явну схему.

    \hrule

    Навести приклади процесів, які моделюються за допомогою диференціальних рівнянь у частинних похідних гіперболічного типу
    
    \newpage
    \section{ОГЛЯД ТА АНАЛІЗ ІСНУЮЧИХ МЕТОДІВ ЧИСЕЛЬНОГО РОЗВ’ЯЗАННЯ ДРЧП}

    \newpage
    \section{ДОСЛІДЖЕННЯ УМОВ ЗАСТОСУВАННЯ ОБРАНОГО МЕТОДУ}
        \begin{enumerate}
            \item 
                Тришарова схема з вагами

                $$ u(t, x) $$
                $$ \frac{\partial^2{u}}{\partial{t^2}} = \frac{\partial^2{u}}{\partial{x^2}} + F(t, x) $$
                
                \begin{align*}
                    \frac{u_i^{k+1} - 2 \cdot u_i^{k} + u_i^{k-1}}{\Delta{t}^2}& = \sigma_1 \cdot \left(\frac{u_{i+1}^{k+1} - 2 \cdot u_i^{k+1} + u_{i-1}^{k+1}}{\Delta{x}^2} + F_i^{k+1}\right) + \\
                    & + \left(1 - \sigma_1 - \sigma_2 \right) \cdot \left(\frac{u_{i+1}^{k} - 2 \cdot u_i^{k} + u_{i-1}^{k}}{\Delta{x}^2} + F_i^{k}\right) + \\
                    & + \sigma_2 \cdot \left(\frac{u_{i+1}^{k-1} - 2 \cdot u_i^{k-1} + u_{i-1}^{k-1}}{\Delta{x}^2} + F_i^{k-1}\right)
                \end{align*}
                
                \begin{align*}
                    u_i^{k+1} & = 2 \cdot u_i^{k} - u_i^{k-1} + \Delta{t}^2 \cdot \sigma_1 \cdot \left(\frac{u_{i+1}^{k+1} - 2 \cdot u_i^{k+1} + u_{i-1}^{k+1}}{\Delta{x}^2} + F_i^{k+1}\right) + \\
                    & + \Delta{t}^2 \cdot \left(1 - \sigma_1 - \sigma_2 \right) \cdot \left(\frac{u_{i+1}^{k} - 2 \cdot u_i^{k} + u_{i-1}^{k}}{\Delta{x}^2} + F_i^{k}\right) + \\
                    & + \Delta{t}^2 \cdot \sigma_2 \cdot \left(\frac{u_{i+1}^{k-1} - 2 \cdot u_i^{k-1} + u_{i-1}^{k-1}}{\Delta{x}^2} + F_i^{k-1}\right)
                \end{align*}
                
                \begin{align*}
                    u_i^{k+1} & - \frac{\Delta{t}^2 \cdot \sigma_1}{\Delta{x}^2} \cdot \left(u_{i+1}^{k+1} - 2 \cdot u_i^{k+1} + u_{i-1}^{k+1}\right) = \\
                    & = 2 \cdot u_i^{k} - u_i^{k-1}  + \\
                    & + \Delta{t}^2 \cdot \sigma_1 \cdot F_i^{k+1} + \\
                    & + \Delta{t}^2 \cdot \left(1 - \sigma_1 - \sigma_2 \right) \cdot \left(\frac{u_{i+1}^{k} - 2 \cdot u_i^{k} + u_{i-1}^{k}}{\Delta{x}^2} + F_i^{k}\right) + \\
                    & + \Delta{t}^2 \cdot \sigma_2 \cdot \left(\frac{u_{i+1}^{k-1} - 2 \cdot u_i^{k-1} + u_{i-1}^{k-1}}{\Delta{x}^2} + F_i^{k-1}\right)
                \end{align*}
                
                \begin{align*}
                    u(k+1, i) & - \frac{\Delta{t}^2 \cdot \sigma_1}{\Delta{x}^2} \cdot \left(u(k+1, i+1) - 2 \cdot u(k+1, i) + u(k+1, i-1)\right) = \\
                    & = 2 \cdot u(k, i) - u(k-1, i)  + \\
                    & + \Delta{t}^2 \cdot \sigma_1 \cdot F(k+1, i) + \\
                    & + \Delta{t}^2 \cdot \left(1 - \sigma_1 - \sigma_2 \right) \cdot \left(\frac{u(k, i+1) - 2 \cdot u(k, i) + u(k, i-1)}{\Delta{x}^2} + F(k, i)\right) + \\
                    & + \Delta{t}^2 \cdot \sigma_2 \cdot \left(\frac{u(k-1, i+1) - 2 \cdot u(k-1, i) + u(k-1, i-1)}{\Delta{x}^2} + F(k-1, i)\right)
                \end{align*}

                \begin{itemize}
                    \item Переваги: Простота реалізації, висока швидкість
                    \item Недоліки:
                        \begin{itemize}
                            \item Для коректної роботи схеми: $ \sigma_1 \geq \sigma_2 $;
                            \item $ \sigma_1 + \sigma_2 \geq \frac{1}{2} $ --- схема є \textbf{стійкою} для будь-яких $ \Delta{x} $ та $ \Delta{t} $;
                            \item $ \sigma_1 + \sigma_2 < \frac{1}{2} $ --- схема є \textbf{умовно стійкою}, тобто вона буде працювати для:
                            $$ \Delta{t} \leq \frac{\Delta{x}}{\sqrt{1 - 2 \cdot(\sigma_1 + \sigma_2)}} $$
                    \end{itemize}
                \end{itemize}

            \item 
                Явна схема

                $$ u(t, x) $$
                $$ \frac{\partial^2{u}}{\partial{t^2}} = \frac{\partial^2{u}}{\partial{x^2}} + F(t, x) $$
                $$ \frac{u_i^{k+1} - 2 \cdot u_i^{k} + u_i^{k-1}}{\Delta{t}^2} = \frac{u_{i+1}^{k} - 2 \cdot u_i^{k} + u_{i-1}^{k}}{\Delta{x}^2} + F_i^{k} $$
                $$ u_i^{k+1} - 2 \cdot u_i^{k} + u_i^{k-1} = \Delta{t}^2 \cdot \left( \frac{u_{i+1}^{k} - 2 \cdot u_i^{k} + u_{i-1}^{k}}{\Delta{x}^2} + F_i^{k} \right) $$
                $$ u_i^{k+1} = 2 \cdot u_i^{k} - u_i^{k-1} + \Delta{t}^2 \cdot \left( \frac{u_{i+1}^{k} - 2 \cdot u_i^{k} + u_{i-1}^{k}}{\Delta{x}^2} + F_i^{k} \right) $$
                $$ u(k+1, i) = 2 \cdot u(k, i) - u(k-1, i) + \Delta{t}^2 \cdot \left( \frac{u(k, i+1) - 2 \cdot u(k, i) + u(k, i-1)}{\Delta{x}^2} + F(k, i) \right) $$

                \begin{itemize}
                    \item Переваги: Простота реалізації, висока швидкість
                    \item Недоліки: Необхідна стійкість $ \frac{\Delta{t}}{\Delta{x}^2} \leq 0.5 $
                \end{itemize}
        \end{enumerate}

    \newpage
    \section{ОПИС ПРОГРАМНОЇ РЕАЛИЗАЦІЇ}
        Параметри:

        Кількість вузлів $ x = 100 $
        
        Кількість індексів дискретного часу $ t = 100 000 $
        
        Відстань між сусідніми просторовими вузлами $ \Delta{x} = 0.01 $
        
        Відстань між сусідніми моментами часу $ \Delta{t} = 0.00005 $

        $ L = 1 $

        Застосовано явну схему.

        Масиви, початковий та кінцевий \eqref{mat:xs}

        Побудовано графіки:

        \begin{figure}[h!]
            \includegraphics[width=0.5\linewidth]{straight_2d.png}
            \includegraphics[width=0.5\linewidth]{straight_3d.png}
            \caption{Поверхня U(t, x) у 2D та 3D}
        \end{figure}

        \begin{figure}[h!]
            \includegraphics[width=0.5\linewidth]{straight_2d.png}
            \includegraphics[width=0.5\linewidth]{straight_3d.png}
            \caption{Зрізи U(t, x) для фіксованих моментів часу}
        \end{figure}

    \newpage
    \section{ОГЛЯД МЕТОДІВ ПІДВИЩЕННЯ ТОЧНОСТІ}

    \newpage
    \section{ЗАСТОСУВАННЯ МЕТОДУ ПІДВИЩЕННЯ ТОЧНОСТІ ТА ЕФЕКТИВНОСТІ РОЗВ’ЯЗКУ ДО ПРИКЛАДУ РОБОТИ}

    \newpage
    \section{ВИСНОВКИ}
    \newpage
    \section{СПИСОК ВИКОРИСТАНИХ ДЖЕРЕЛ}
    \begin{enumerate}
        \item 
    \end{enumerate}
    \newpage
    \section{ДОДАТКИ}
        Різне

        Start Matrix (100000x100): 
        $$
        \left(\begin{matrix}
            0.0 & 0.01020304050607081 & 0.020610141822263034 & $\dots$ & 1.9697990001020307 & 2.0 \\
            0.0 & 0.0 & 0.0 & $\dots$ & 0.0 & 1.9999999753254956 \\
            0.0 & 0.0 & 0.0 & $\dots$ & 0.0 & 1.999999901301983 \\
            0.0 & 0.0 & 0.0 & $\dots$ & 0.0 & 1.9999997779294636 \\
            0.0 & 0.0 & 0.0 & $\dots$ & 0.0 & 1.9999996052079412 \\
            0.0 & 0.0 & 0.0 & $\dots$ & 0.0 & 1.9999993831374194 \\
            0.0 & 0.0 & 0.0 & $\dots$ & 0.0 & 1.9999991117179041 \\
            0.0 & 0.0 & 0.0 & $\dots$ & 0.0 & 1.9999987909494017 \\
            0.0 & 0.0 & 0.0 & $\dots$ & 0.0 & 1.9999984208319204 \\
            0.0 & 0.0 & 0.0 & $\dots$ & 0.0 & 1.9999980013654692 \\

            \dots & \dots & \dots & \dots & \dots & \dots \\

            0.0 & 0.0 & 0.0 & $\dots$ & 0.0 & -1.999999901301983 \\
            0.0 & 0.0 & 0.0 & $\dots$ & 0.0 & -1.9999999753254956 \\
            0.0 & 0.0 & 0.0 & $\dots$ & 0.0 & -2.0 \\
            \label{mat:xs}
        \end{matrix}\right)
        $$
    
        Received Matrix (100000x100): 
        $$
        \left(\begin{matrix}
            0.0 & 0.01020304050607081 & 0.020610141822263034 & $\dots$ & 1.9697990001020307 & 2.0 \\
            0.0 & -1.9999999805773505 & -3.9999507106768935 & $\dots$ & -188.43500752856792 & 1.9999999753254956 \\
            0.0 & -1.9999999065538376 & -3.9999998186151107 & $\dots$ & -195.91107678311408 & 1.999999901301983 \\
            0.0 & -1.9999997831813179 & -3.999999571870071 & $\dots$ & -195.99956337495996 & 1.9999997779294636 \\
            0.0 & -1.999999610459795 & -3.999999226427025 & $\dots$ & -196.0000441955736 & 1.9999996052079412 \\
            0.0 & -1.9999993883892728 & -3.9999987822859797 & $\dots$ & -196.00002406230797 & 1.9999993831374194 \\
            0.0 & -1.9999991169697569 & -3.999998239446947 & $\dots$ & -195.9999974666734 & 1.9999991117179041 \\
            0.0 & -1.9999987962012535 & -3.99999759790994 & $\dots$ & -195.99996603135077 & 1.9999987909494017 \\
            0.0 & -1.999998426083771 & -3.9999968576749736 & $\dots$ & -195.9999297598208 & 1.9999984208319204 \\
            0.0 & -1.9999980066173189 & -3.9999960187420682 & $\dots$ & -195.99988865209056 & 1.9999980013654692 \\

            \dots & \dots & \dots & \dots & \dots & \dots \\

            0.0 & 1.9999999065538376 & 3.9999998186151107 & $\dots$ & 195.9116714171308 & -1.999999901301983 \\
            0.0 & 1.9999999805773505 & 3.9999509606818933 & $\dots$ & 188.4918441248053 & -1.9999999753254956 \\
            0.0 & 2.000000005251855 & 3.9999512600365073 & $\dots$ & 188.54868323177544 & -2.0 \\
            \label{mat:xr}
        \end{matrix}\right)
        $$

        % \includegraphics{}
        % \includegraphics{}
        % \includegraphics{}
        % \includegraphics{}

\end{document}