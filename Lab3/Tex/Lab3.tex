\documentclass{article}
\usepackage[utf8]{inputenc}
\usepackage[english, ukrainian]{babel}
\usepackage{fontsize}
\usepackage{geometry}
\usepackage{amsthm}
\usepackage{amsfonts}
\usepackage{graphicx}
\usepackage[ruled]{algorithm2e}
\usepackage{hyperref}
\usepackage{biblatex}
\usepackage{csquotes}
\usepackage{mathtools}
\usepackage{amsmath}
\usepackage{amssymb}
\usepackage{bbm}
\usepackage{tabularx}
\hypersetup{colorlinks=true, linkcolor=[RGB]{255, 3, 209}, citecolor={black}}

\graphicspath{ {../Images/} }

\begin{document}
    \begin{titlepage}
        \begin{center}
            \begin{center}
                НАЦІОНАЛЬНИЙ ТЕХНІЧНИЙ УНІВЕРСИТЕТ УКРАЇНИ
                «КИЇВСЬКИЙ ПОЛІТЕХНІЧНИЙ ІНСТИТУТ імені Ігоря СІКОРСЬКОГО»

                Фізико-технічний інститут
            \end{center}
        $\newline$
        \vspace{3.3cm}
        
        {КОМП’ЮТЕРНИЙ ПРАКТИКУМ № 3.\\РОЗВ’ЯЗАННЯ СЛАР ІТЕРАЦІЙНИМИ МЕТОДАМИ}
        \vspace{5cm}
        \begin{flushright}
            Виконав\\студент 3 курсу ФТІ\\групи ФІ-21\\Климентьєв Максим Андрійович
            
            \vspace{1cm}

            Перевірив:\\\underline{\hspace{5cm}}\\Оцінка:\\\underline{\hspace{5cm}}
        \end{flushright}
        \vspace{3cm}
        Київ --- 2025
        \end{center}
    \end{titlepage}
    \newpage

    \pagenumbering{gobble}
    \tableofcontents
    \cleardoublepage
    \pagenumbering{arabic}
    \setcounter{page}{3}

    % Для методу простої ітерації навести письмовий етап приведення матриці до діагональної
    % переваги; лістинг програми та результати її дії (на кожній ітерації вивести номер ітерації,
    % наближене значення розв'язку, значення критерію).

    % Таким чином, звіт має містити:
    % - письмовий етап приведення матриці до діагональної переваги,
    % - результати перших трьох та останньої ітерацій методу,
    % - вектор нев’язки на кожній ітерації,
    % - лістинг програми.

    \newpage

    \section{Дані}
    \begin{tabular}{ |c|c|c| }
        \hline
        Варіант & Матриця системи А & Вектор правої частини b \\ 
        \hline
        10
        &
        $ \begin{matrix}
            6.59 & 1.28 & 0.79 & 1.195 & -0.21\\
            0.92 & 3.83 & 1.3 & -1.63 & 1.02\\
            1.15 & -2.46 & 5.77 & 2.1 & 1.483\\
            1.285 & 0.16 & 2.1 & 5.77 & -18\\
            0.69 & -1.68 & -1.217 & 9 & -6
        \end{matrix} $
        &
        $ \begin{matrix}
            2.1\\
            0.36\\
            3.89\\
            11.04\\
            -0.27
        \end{matrix} $
        \\ 
        \hline
    \end{tabular}

    \section{Письмовий етап приведення матриці до діагональної переваги}

    \foreach \x in {1, 2, ..., 6}
    {
        \includegraphics[width=\textwidth]{\x.png}
    }

    \section{Результати перших трьох та останньої ітерацій методу}

    
$$ [0] x: \left(\begin{matrix}
    0.31866464 \\
    0.08577713 \\
    0.62802668 \\
    -0.79021017 \\
    -0.56835404
\end{matrix}\right) $$

    
$$ [1] x: \left(\begin{matrix}
    0.35189848 \\
    -0.1360347  \\
    0.93382286 \\
    -0.39921256 \\
    -0.80300943
\end{matrix}\right) $$   

    
$$ [2] x: \left(\begin{matrix}
    0.27994407 \\
    -0.34009852 \\
    0.81732981 \\
    -0.29008047 \\
    -0.61228626
\end{matrix}\right) $$   

    
$$ [15] x: \left(\begin{matrix}
    0.32955216 \\
    -0.26880975 \\
    0.80811961 \\
    -0.41730668 \\
    -0.63169572

\end{matrix}\right) $$
    \section{Вектор нев’язки на кожній ітерації}
    
    
$$ [0] r: \left(\begin{matrix}
    0.21901101 \\
    0.75637832 \\
    2.00804118 \\
    5.76721468 \\
    1.81341687
\end{matrix}\right) $$
   
   
   
$$ [1] r: \left(\begin{matrix}
    0.47417961 \\
    0.69585762 \\
    0.76496323 \\
    1.60969837 \\
    1.47390867
\end{matrix}\right) $$
   
   
   
 $$[2] r: \left(\begin{matrix}
    0.26287021 \\
    0.21849338 \\
    0.29258683 \\
    2.18535873 \\
    0.22868634
\end{matrix}\right) $$
   
   
   
$$ [3] r: \left(\begin{matrix}
    0.13645018 \\
    0.10685171 \\
    0.25419385 \\
    0.10085488 \\
    0.47439844
\end{matrix}\right) $$   
   
$$ [4] r: \left(\begin{matrix}
    0.07540986 \\
    0.08639293 \\
    0.03490878 \\
    0.57296527 \\
    0.01644702
\end{matrix}\right) $$
   
   
$$ [5] r: \left(\begin{matrix}
    0.01774368 \\
    0.01585437 \\
    0.07323021 \\
    0.08107878 \\
    0.11986356
\end{matrix}\right) $$
   
    
$$ [6] r:\left(\begin{matrix}
    0.02458714 \\
    0.0248991  \\
    0.00391974 \\
    0.1391443  \\
    0.02637121
\end{matrix}\right) $$
   
   
$$ [7] r: \left(\begin{matrix}
    0.00073858 \\
    0.00045966 \\
    0.01873128 \\
    0.0457133  \\
    0.02727432
\end{matrix}\right) $$
   
   
   
 $$ [8] r: \left(\begin{matrix}
    0.00652564 \\
    0.00657839 \\
    0.00435757 \\
    0.0285816  \\
    0.01177387
\end{matrix}\right) $$
   
   
   
$$ [9] r: \left(\begin{matrix}
    0.0009979  \\
    0.00127251 \\
    0.00418348 \\
    0.01722922 \\
    0.00511326
\end{matrix}\right) $$
   
   
   
$$ [10] r: \left(\begin{matrix}
    0.00156045 \\
    0.00151075 \\
    0.00189869 \\
    0.00451853 \\
    0.00403224
\end{matrix}\right) $$
   
   
$$ [11] r: \left(\begin{matrix}
    0.000539   \\
    0.00060713 \\
    0.00077069 \\
    0.0053632  \\
    0.0006388 
\end{matrix}\right) $$
   
   
$$ [12] r: \left(\begin{matrix}
    0.00031669 \\
    0.00029262 \\
    0.00063982 \\
    0.00024233 \\
    0.00118112
\end{matrix}\right) $$
   
   
$$ [13] r: \left(\begin{matrix}
    1.99277455e-04 \\
    2.12698169e-04 \\
    9.03525690e-05 \\
    1.46380418e-03 \\
    4.09051201e-05
\end{matrix}\right) $$   

   
$$ [14] r: \left(\begin{matrix}
    4.85115522e-05 \\
    3.96531294e-05 \\
    1.85525718e-04 \\
    1.95977908e-04 \\
    3.05581645e-04
\end{matrix}\right) $$   

     
$$ [15] r:\left(\begin{matrix}
    6.13856585e-05 \\
    6.33931185e-05 \\
    9.05118633e-06 \\
    3.52829061e-04 \\
    6.51510711e-05
\end{matrix}\right) $$

    \section{Лістинг програми}
    \begin{lstlisting}[language=Python, caption=Simple Iteration Method]
def Iteration_simple_method(C, d, epsilon=1e-4):
    xk = np.zeros(b.shape)
    xk1 = C@xk + d

    count = 0

    r = abs(b_start - A_start@xk1)
    print(f"[{count}] r: \n {r} \n")
    print(f"[{count}] x: \n {xk1} \n\n")
    
    q = np.zeros(b.shape)
    for row in range(C.shape[0]):
        q = min(sum(abs(C[row])), sum(abs(C[:, row])))

    while q/(1-q) * np.max(abs(xk1 - xk)) >= epsilon:
        xk = xk1
        xk1 = C@xk + d
    
        count += 1
    
        r = abs(b_start - A_start@xk1)
        print(f"[{count}] r: \n {r} \n")
        print(f"[{count}] x: \n {xk1} \n\n")

    return xk1
    \end{lstlisting}
\end{document}