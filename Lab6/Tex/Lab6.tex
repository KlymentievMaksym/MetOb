\documentclass{article}
\usepackage[utf8]{inputenc}
\usepackage[english, ukrainian]{babel}
\usepackage{fontsize}
\usepackage{geometry}
\usepackage{amsthm}
\usepackage{amsfonts}
\usepackage{graphicx}
\usepackage[ruled]{algorithm2e}
\usepackage{hyperref}
\usepackage{biblatex}
\usepackage{csquotes}
\usepackage{mathtools}
\usepackage{amsmath}
\usepackage{amssymb}
\usepackage{bbm}
\usepackage{tabularx}
\hypersetup{colorlinks=true, linkcolor=[RGB]{255, 3, 209}, citecolor={black}}

\graphicspath{ {../Images/} }

\begin{document}
    \begin{titlepage}
        \begin{center}
            \begin{center}
                НАЦІОНАЛЬНИЙ ТЕХНІЧНИЙ УНІВЕРСИТЕТ УКРАЇНИ
                «КИЇВСЬКИЙ ПОЛІТЕХНІЧНИЙ ІНСТИТУТ імені Ігоря СІКОРСЬКОГО»

                Фізико-технічний інститут
            \end{center}
        $\newline$
        \vspace{3.3cm}
        
        {КОМП’ЮТЕРНИЙ ПРАКТИКУМ № 6.\\РОЗВ’ЯЗАННЯ ЗАДАЧІ КОШІ МЕТОДАМИ РУНГЕ-КУТТА ТА АДАМСА}
        \vspace{5cm}
        \begin{flushright}
            Виконав\\студент 3 курсу ФТІ\\групи ФІ-21\\Климентьєв Максим Андрійович
            
            \vspace{1cm}

            Перевірив:\\\underline{\hspace{5cm}}\\Оцінка:\\\underline{\hspace{5cm}}
        \end{flushright}
        \vspace{3cm}
        Київ --- 2025
        \end{center}
    \end{titlepage}
    \newpage

    \pagenumbering{gobble}
    \tableofcontents
    \cleardoublepage
    \pagenumbering{arabic}
    \setcounter{page}{3}

    \newpage
    \section{ПОСТАНОВКА ЗАДАЧІ}
    ПОСТАНОВКА ЗАДАЧІ:

        Рівняння має вигляд:
        $y' = (1 - y) \cdot x^2 + F(x)$
        Покласти $h = 0.1$. Початкові умови $y(0)$ визначити, використовуючи точне значення розв’язку.

        Нехай розв’язок відомий та визначається згідно з варіантами:

    \section{Вихідна система}
        \begin{tabular}{ |c|c| }
            \hline
            Варіант & Точний розв’язок \\ 
            \hline
            10 & $y = x \cdot \sin(x)$\\ 
            \hline
        \end{tabular}
\end{document}