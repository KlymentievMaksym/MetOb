% Звіт
% 1. Вихідні дані.

% 2. Письмове виконання допрограмового етапу, результатом якого повинні бути проміжки, щодо яких проводиться уточнення.

% 3. Лістинг програми уточнення коренів за методами бісекції, хорд, дотичних
% (вхідними даними для цієї програми є координати проміжків [ai, bi] та коефіцієнти поліному)
% та результати дії програми. На кожній ітерації методу слід виводити такі дані: номер ітерації, наближене значення кореня, критерій завершення ітерацій.

\documentclass{article}
\usepackage{graphicx}
\usepackage{epstopdf}
\usepackage{lipsum}
\usepackage[T2A]{fontenc}
\usepackage[utf8]{inputenc}

\graphicspath{ {../Images/} }

\begin{document}
    \begin{titlepage}
        \begin{center}
        $\newline$
        \vspace{3.3cm}
        
        {\LARGE\textbf{КОМП’ЮТЕРНИЙ ПРАКТИКУМ № 1.\\ РОЗВ’ЯЗАННЯ НЕЛІНІЙНИХ РІВНЯНЬ"}}
        \vspace{10cm}
        \begin{flushright}
            \textbf{Роботу виконав:}\\Климентьєв Максим \\3-го курсу\\групи ФІ-21
        \end{flushright}
        \end{center}
    \end{titlepage}
    \newpage

    \tableofcontents
    \newpage
    \section{Вихідні дані.}
    \begin{tabular}{ |c|c| }
        \hline
        Варіант & Вигляд рівняння \\ 
        \hline
        10 & $ -2 \cdot x^4 + x^3 + 5 \cdot x^2 - 2 \cdot x + 7 = 0 $ \\ 
        \hline
    \end{tabular}

    \section{Письмове виконання допрограмового етапу, результатом якого повинні бути проміжки, щодо яких проводиться уточнення.}
    % \includegraphics{}
    % \includegraphics{}
    % \includegraphics{}
    \newpage
    \section{Лістинг програми уточнення коренів за методами бісекції, хорд, дотичних (вхідними даними для цієї програми є координати проміжків [ai, bi] та коефіцієнти поліному) та результати дії програми. На кожній ітерації методу слід виводити такі дані: номер ітерації, наближене значення кореня, критерій завершення ітерацій.}
        \textbf{} --- 
        
\end{document}