\documentclass{article}
\usepackage[utf8]{inputenc}
\usepackage[english, ukrainian]{babel}
\usepackage{fontsize}
\usepackage{geometry}
\usepackage{amsthm}
\usepackage{amsfonts}
\usepackage{graphicx}
\usepackage[ruled]{algorithm2e}
\usepackage{hyperref}
\usepackage{biblatex}
\usepackage{csquotes}
\usepackage{mathtools}
\usepackage{amsmath}
\usepackage{amssymb}
\usepackage{bbm}
\usepackage{tabularx}
\usepackage{xcolor}

\usepackage{tikz}
\usetikzlibrary{decorations.pathmorphing}

\usepackage{enumitem}
\usepackage{nicefrac}

\usepackage{listings}
\definecolor{codegreen}{rgb}{0,0.6,0}
\definecolor{codegray}{rgb}{0.5,0.5,0.5}
\definecolor{codepurple}{rgb}{0.58,0,0.82}
\definecolor{backcolour}{rgb}{0.95,0.95,0.92}

\lstdefinestyle{mystyle}{
    backgroundcolor=\color{backcolour},   
    commentstyle=\color{codegreen},
    keywordstyle=\color{magenta},
    numberstyle=\tiny\color{codegray},
    stringstyle=\color{codepurple},
    basicstyle=\ttfamily\footnotesize,
    breakatwhitespace=false,         
    breaklines=true,                 
    captionpos=b,                    
    keepspaces=true,                 
    numbers=left,                    
    numbersep=5pt,                  
    showspaces=false,                
    showstringspaces=false,
    showtabs=false,                  
    tabsize=2
}

\lstset{style=mystyle}
\hypersetup{colorlinks=true, linkcolor=[RGB]{255, 3, 209}, citecolor={black}}

\graphicspath{ {../Images/} }

\begin{document}
    \begin{titlepage}
        \begin{center}
            \begin{center}
                НАЦІОНАЛЬНИЙ ТЕХНІЧНИЙ УНІВЕРСИТЕТ УКРАЇНИ
                «КИЇВСЬКИЙ ПОЛІТЕХНІЧНИЙ ІНСТИТУТ імені Ігоря СІКОРСЬКОГО»

                Фізико-технічний інститут
            \end{center}
        $\newline$
        \vspace{3.3cm}
        
        {КОМП’ЮТЕРНИЙ ПРАКТИКУМ № 4.\\ОБЧИСЛЕННЯ ВЛАСНИХ ЗНАЧЕНЬ}
        \vspace{5cm}
        \begin{flushright}
            Виконав\\студент 3 курсу ФТІ\\групи ФІ-21\\Климентьєв Максим Андрійович
            
            \vspace{1cm}

            Перевірив:\\\underline{\hspace{5cm}}\\Оцінка:\\\underline{\hspace{5cm}}
        \end{flushright}
        \vspace{3cm}
        Київ --- 2025
        \end{center}
    \end{titlepage}
    \newpage

    \pagenumbering{gobble}
    \tableofcontents
    \cleardoublepage
    \pagenumbering{arabic}
    \setcounter{page}{3}

    \newpage
    \section{ПОСТАНОВКА ЗАДАЧІ}

    Звіт має містити:

    - Для методу Данилевського: матриці $M_i$ та $M_i^-1$ ; результуючу матрицю у формі Фробеніуса ($P$), отримане характеристичне рівняння (коефіцієнти полінома), власні числа ($\lambda$).

    Для всіх варіантів у звіті потрібно навести перевірку у Маthcad чи Matlab

    Для методу Данилевського: привести матрицю до вигляду Фробеніуса, розв’язати отриману систему за допомогою методу із практикуму 2 або 3, отримати коефіцієнти характеристичного рівняння. Розв’язати характеристичне рівняння за допомогою одного з методів із практикуму 1 і отримати власні числа.
    \section{Вихідна система}
    \begin{tabular}{ |c|c|c| }
        \hline
        Варіант & Матриця & Метод \\ 
        \hline
        10
        &
        $\begin{matrix}
            6.20 & 1.10 & 0.94 & 1.21\\
            1.10 & 4.10 & 1.30 & 0.16\\
            0.94 & 1.30 & 7.40 & 1.10\\
            1.21 & 0.16 & 1.10 & 9.10
        \end{matrix}$
        &
        Данілевського
        \\ 
        \hline
    \end{tabular}

    \section{}
    $A_3 = $
    $
    \left(\begin{matrix}
        6.2 & 1.1 & 0.94 & 1.21 \\
        1.1 & 4.1 & 1.3 & 0.16 \\
        0.94 & 1.3 & 7.4 & 1.1 \\
        1.21 & 0.16 & 1.1 & 9.1 \\
    \end{matrix}\right)
    $

    $M^{-1}_3 = $
    $
    \left(\begin{matrix}
        1.0 & 0.0 & 0.0 & 0.0 \\
        0.0 & 1.0 & 0.0 & 0.0 \\
        1.21 & 0.16 & 1.1 & 9.1 \\
        0.0 & 0.0 & 0.0 & 1.0 \\
    \end{matrix}\right)
    $

    $M_3 = $
    $
    \left(\begin{matrix}
        1.0 & 0.0 & 0.0 & 0.0 \\
        0.0 & 1.0 & 0.0 & 0.0 \\
        -1.1 & -0.1455 & 0.9091 & -8.2727 \\
        0.0 & 0.0 & 0.0 & 1.0 \\
    \end{matrix}\right)
    $

    $A_2 = $
    $
    \left(\begin{matrix}
        5.166 & 0.9633 & 0.8545 & -6.5664 \\
        -0.33 & 3.9109 & 1.1818 & -10.5945 \\
        -1.7219 & 2.0373 & 17.7231 & -75.7704 \\
        0.0 & -0.0 & 1.0 & 0.0 \\
    \end{matrix}\right)
    $

    $M^{-1}_2 = $
    $
    \left(\begin{matrix}
        1.0 & 0.0 & 0.0 & 0.0 \\
        -1.7219 & 2.0373 & 17.7231 & -75.7704 \\
        0.0 & 0.0 & 1.0 & 0.0 \\
        0.0 & 0.0 & 0.0 & 1.0 \\
    \end{matrix}\right)
    $

    $M_2 = $
    $
    \left(\begin{matrix}
        1.0 & 0.0 & 0.0 & 0.0 \\
        0.8452 & 0.4908 & -8.6993 & 37.1915 \\
        0.0 & 0.0 & 1.0 & 0.0 \\
        0.0 & 0.0 & 0.0 & 1.0 \\
    \end{matrix}\right)
    $

    $A_1 = $
    $
    \left(\begin{matrix}
        5.9802 & 0.4728 & -7.5252 & 29.2592 \\
        -4.2354 & 20.8198 & -129.7181 & 224.3644 \\
        -0.0 & 1.0 & 0.0 & 0.0 \\
        0.0 & -0.0 & 1.0 & -0.0 \\
    \end{matrix}\right)
    $

    $M^{-1}_1 = $
    $
    \left(\begin{matrix}
        -4.2354 & 20.8198 & -129.7181 & 224.3644 \\
        0.0 & 1.0 & 0.0 & 0.0 \\
        0.0 & 0.0 & 1.0 & 0.0 \\
        0.0 & 0.0 & 0.0 & 1.0 \\
    \end{matrix}\right)
    $

    $M_1 = $
    $
    \left(\begin{matrix}
        -0.2361 & 4.9156 & -30.6268 & 52.9731 \\
        0.0 & 1.0 & 0.0 & 0.0 \\
        0.0 & 0.0 & 1.0 & 0.0 \\
        0.0 & 0.0 & 0.0 & 1.0 \\
    \end{matrix}\right)
    $

    $A_0$
    $
    \left(\begin{matrix}
        26.8 & -256.2267 & 1031.9724 & -1465.6609 \\
        1.0 & 0.0 & 0.0 & 0.0 \\
        0.0 & 1.0 & 0.0 & -0.0 \\
        -0.0 & 0.0 & 1.0 & 0.0 \\
    \end{matrix}\right)
    $

    $P = M^{-1} \cdot Matrix \cdot M = $
    $
    \left(\begin{matrix}
        26.8 & -256.2267 & 1031.97237 & -1465.660941 \\
        1.0 & 0.0 & -0.0 & 0.0 \\
        -0.0 & 1.0 & -0.0 & 0.0 \\
        -0.0 & 0.0 & 1.0 & -0.0 \\
    \end{matrix}\right)
    $

    $ Coefs = $
    $
    \left(\begin{matrix}
        26.8 \\
        -256.2267 \\
        1031.9724 \\
        -1465.6609 \\
    \end{matrix}\right)
    $

    $ \lambda = $
    $
    \left(\begin{matrix}
        3.3875 \\
        5.6635 \\
        7.3377 \\
        10.4112 \\
    \end{matrix}\right)
    $

    \section{Перевірка через Numpy}

    $
    check = np.linalg.eigvals(matrix)\\
    check = np.atleast\_2d(check).T\\
    print(latex\_matrix(np.round(check, 4)))
    $

    $
    \left(\begin{matrix}
        10.4112 \\
        3.3875 \\
        5.6635 \\
        7.3377 \\
    \end{matrix}\right)
    $

\end{document}